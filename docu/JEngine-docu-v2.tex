%%%%%%%%%%%%%%%%%%%%%%%%%%%%%%%%%%%%%%%%%
% University/School Laboratory Report
% LaTeX Template
% Version 3.1 (25/3/14)
%
% This template has been downloaded from:
% http://www.LaTeXTemplates.com
%
% Original author:
% Linux and Unix Users Group at Virginia Tech Wiki 
% (https://vtluug.org/wiki/Example_LaTeX_chem_lab_report)
%
% License:
% CC BY-NC-SA 3.0 (http://creativecommons.org/licenses/by-nc-sa/3.0/)
%
%%%%%%%%%%%%%%%%%%%%%%%%%%%%%%%%%%%%%%%%%

%----------------------------------------------------------------------------------------
%	PACKAGES AND DOCUMENT CONFIGURATIONS
%----------------------------------------------------------------------------------------

% please be aware to install sudo apt-get install texlive-science

\documentclass{article}

%\usepackage[version=3]{mhchem} % Package for chemical equation typesetting
\usepackage{siunitx} % Provides the \SI{}{} and \si{} command for typesetting SI units
\usepackage{graphicx} % Required for the inclusion of images
\usepackage{natbib} % Required to change bibliography style to APA
\usepackage{amsmath} % Required for some math elements 
\usepackage{fullpage}

\setlength\parindent{0pt} % Removes all indentation from paragraphs

\renewcommand{\labelenumi}{\alph{enumi}.} % Make numbering in the enumerate environment by letter rather than number (e.g. section 6)

%\usepackage{times} % Uncomment to use the Times New Roman font

%----------------------------------------------------------------------------------------
%	DOCUMENT INFORMATION
%----------------------------------------------------------------------------------------

\title{Documentation zur JEngine} % Title

\author{BP2014W1 Team} % Author name

\date{\today} % Date for the report

\begin{document}

\maketitle % Insert the title, author and date

\begin{center}
\begin{tabular}{l r}
Date Performed: & January 1, 2015 \\ % Date the experiment was performed
%Betreuer: & Andreas Meyer \\ % Partner names
%& Marcin Hewelt \\
%Instructor: & Professor Dr Mathias Weske % Instructor/supervisor
\end{tabular}
\end{center}

%----------------------------------------------------------------------------------------
%	SECTION 0 
%----------------------------------------------------------------------------------------

% If you wish to include an abstract, uncomment the lines below
\begin{abstract}
Diese Dokumentation ist entstanden im Rahmen des Bachelorprojekts BP2014W1 am Lehrstuhl für ``Business Process Technology'' betreut durch Prof. Dr. Mathias Weske. Es dient zu Dokumentierung der konzipierten und implementierten JEngine um ein Proof-of-Concept zu ermöglichen und gleichzeitig als Prototype für Anwendungsfälle von Bosch Software Innovations zu fungieren.
\end{abstract}

%\tableofcontents
%\listoffigures
%\listoftables
%\lstlistoflistings
%\listofsymbols{ll}{$w$ & The weight vector}
%\acknowledgements{Thanks to no one.}
%\dedicatory{To \dots}
%\mainmatter

%----------------------------------------------------------------------------------------
%	SECTION 1
%----------------------------------------------------------------------------------------


%
%
\section{Introduction}
Productive Case Management (PCM) beschreibt eine.... \cite{ImplementationFrameworkPCM} (siehe Abbildung \ref{fig:PCMmetaModell}).

%
%
\section{MetaModell}

\begin{figure}
\centering
\includegraphics[width=3in]{img/MetaModell_classDiagramm.jpg}
\caption{Meta Modell von PCM.}
\label{fig:PCMmetaModell}
\end{figure}

%
%
\section{JEngine}
Overall JEngine

\begin{figure}
\centering
\includegraphics[width=3in]{img/JEngine_overview.png}
\caption{Meta Modell von PCM.}
\label{fig:PCMmetaModell}
\end{figure}


%
%
\subsection{JCore}
Der JCore umfasst mehrere Hauptkomponenten unserer Engine. Dazu zählt die Auswertung der Datenbank und das Entscheiden von enableden Aktivitäten etc.\\
Dazu zählt zum Beispiel auch die REST-API sowie der ExecutionService.\\
In der JCore liegen die grundsächlichsten Funktionalitäten, die unsere JEngine anbietet. Zu diesen gehören zum einen die Möglichkeit, dass man Aktivitäten sequenziell ausführen kann. Bei den Aktivitäten, die man ausführen kann, handelt es sich bisher nur um User-Tasks, die später noch mit E-Mail-Tasks erweitert werden sollen.\\
Bei PCM können Aktivitäten aus Datenfluss- und Datenobjektflusssicht enabled werden. Dies ist auch der Grund, weshalb wir Datenobjekte unterstützen, sowie deren Zustandsübergänge. Außerdem können in PCM Aktivitäten referenziert werden, das bei der Ausführung die Folge hat, dass wenn referenzierte Aktivitäten enabled sind und dieselben Vor- und Nachbedngungen haben, auch beide in dne Zustand running wechseln, sobald eine von ihnen gestartet wird.\\
Darüberhinaus bestehen PCM-Fragmente aus einem Subset von BPMN. Davon unterstützen wir zur Zeit Start-/Endevents, Aktivitäten und AND-Gateways.

%
%
\subsubsection{REST-API}
Um eine Kommunikation zwischen unseren verschieden Elementen der Engine und des Front-Ends zu ermöglichen, haben wir uns für ein REST-Interface entschieden. Dabei unterstützen wir bisher 2 Methoden: GET und POST.
\begin{enumerate}
%hier beginnt Aufzählung unserer REST Methoden
\item \textbf{@GET:}\\
Die GET-Requests existieren, um den Nutzern der Engine über ein User Interface eine Möglichkeit zu bieten, einzusehen welche Aktivitäten noch zu bearbeiten sind (also offen sind) bzw. welche Aktivitäten zu welcher Scenarioinstanz gehören und ähnliches. \\

1.1.\\ Um offene bzw. geschlossene Aktivitäten einer ScenarioInstanz ausgegeben zu bekommen, muss ein GET mit folgender URL ausgeführt werden:\\
\textbf{\textit{http://172.16.64.113:8080/JEngine/Scenario/\\\{ScenarioID\}/\{ScenarioInstanceID\}/\{Status\}}}\\

\textbf{Variablen die benutzt werden:}\\
\textit{ScenarioID:} hier wird die ID des Scenarios erwartet. Dabei handelt es sich um einen Integer-Wert.\\
\textit{ScenarioInstanceID:} hier wird die ID der ScenarioInstanz erwartet. Dabei handelt es sich um einen Integer-Wert.\\
\textit{Status:} Der Status ist vom Typ String und muss ein Element der Menge\textbf{\textit{\{terminated, enabled\}}} sein.\\

\textbf{Dabei können folgende Fehler geworfen werden:}\\
\textit{Error: not a correct scenario instance}\\ Dies bedeutet das eine ScenarioInstanzID angegeben worden ist, die nicht existiert.\\
\textit{Error: status not clear}\\ Dieser Fehler besagt, dass ein Status angegeben wurde, der nicht der Menge \textbf{\textit{\{terminated, enabled\}}} entspricht.\\

1.2. \\Wenn man wissen möchte, welche Szenarien man in der JEngine starten kann, erhält man diese über folgende GET URL:\\
\textbf{\textit{http://172.16.64.113:8080/JEngine/Scenario/\\Show}}\\

1.3.\\ Um alle ScenarioInstanzen eines Szenarios zu bekommen, muss ein GET-Request mit folgender URL ausgeführt werden:\\
\textbf{\textit{http://172.16.64.113:8080/JEngine/Scenario/\\Instances/\{ScenarioID\}}}\\

\textbf{Variable die benutzt wird:}\\
\textit{ScenarioID:} Dabei handelt es sich um einen Integer, der die ID des zu betrachtenden Szenarios angibt.\\

\textbf{Dabei kann folgender Fehler geworfen werden:}\\
\textit{Error: not a correct scenario}\\ Dieser tritt auf, wenn eine ID übergeben wurde, die nicht existiert.\\

1.4.\\ Wenn man alle Datenobjekte in ihren entsprechenden Zuständen anzeigen möchte, bezogen auf eine ScenarioInstanz, muss ein GET-Request mit folgender URL ausgfeührt werden:\\
\textbf{\textit{http://172.16.64.113:8080/JEngine/Scenario/\\DataObjects/\{ScenarioID\}/\{ScenarioInstanceID\}}}\\

\textbf{Variablen die benutzt werden:}\\
\textit{ScenarioID:} hier wird die ID des Scenarios erwartet. Dabei handelt es sich um einen Integer-Wert.\\
\textit{ScenarioInstanceID:} hier wird die ID der ScenarioInstanz erwartet. Dabei handelt es sich um einen Integer-Wert.\\

\textbf{Dabei kann folgender Fehler geworfen werden:}\\
\textit{Error: not a correct scenario instance}\\ Wenn eine falsche ScenarioInstanzID angegeben wird, produziert es diesen Fehler.\\

1.5.\\ Um von einer SzenrioInstanzID die dazugehörige ScenrioID zu erhalten, kann ein GET-Request mit folgender URL ausgeführt werden:\\ 
\textbf{\textit{http://172.16.64.113:8080/JEngine/Scenario/\\Get/ScenarioID/\{ScenarioInstanceID\}}}\\

\textbf{Variable die benutzt wird:}\\
\textit{ScenarioInstanceID:} hier wird die ID der ScenarioInstanz erwartet. Dabei handelt es sich um einen Integer-Wert.\\

\textbf{Dabei kann folgender Fehler geworfen werden:}\\
\textit{Error: not a correct scenario instance} \\ Wurde eine ScenarioInstanzID angegeben, die nicht existiert, wird dieser Fehler geworfen.\\

1.6.\\ Wenn man eine AktivitätsInstanzId besitzt, und das dazugehörige Label wissen möchte, kann man einen GET-Request mit folgender URL ausführen:\\
\textit{\textbf{http://172.16.64.113:8080/JEngine/Scenario/\\ActivityID/\{Activity\}}}

\textbf{Variable die benutzt wird:}\\
\textit{Activity:} hier wird die ID der AktivitätsInstanz erwartet. Dabei handelt es sich um einen Integer-Wert.\\

\textbf{Dabei kann folgender Fehler geworfen werden:}\\
\textit{Error: not correct Activity ID}\\ Wurde eine AktivitätsInstanzID angegeben, die nicht existiert, wird dieser Fehler geworfen.\\

\item \textbf{@POST:}\\
Über POST-Request ist es möglich, Aktivitäten zu starten sowie gestartete Aktivitäten zu terminieren. Außerdem besteht auch die Möglichkeit ganze Szenarien zu starten, falls dies benötigt wird. All dies soll mit der Funktion ausgestattet sein, dass man dazu noch einen Kommentar abgeben kann.\\

2.1.\\ Um über einen POST-Request Aktivitäten zu starten bzw. zu beenden, wird die folgende URL genutzt:\\
\textit{\textbf{http://172.16.64.113:8080/JEngine/Scenario/\\\{ScenarioID\}/\{ScenarioInstanceID\}/\\\{ActivityID\}/\{Status\}/comment}}\\

\textbf{Variablen die benutzt werden:}\\
\textit{ScenarioID:} hier wird die ID des Scenarios erwartet. Dabei handelt es sich um einen Integer-Wert.\\
\textit{ScenarioInstanceID:} hier wird die ID der ScenarioInstanz erwartet. Dabei handelt es sich um einen Integer-Wert.\\
\textit{ActivityID:} hier wird die ID der AktivitätsInstanz erwartet. Dabei handelt es sich um einen Integer-Wert.\\
\textit{Status:} Der Status ist vom Typ String und muss ein Element der Menge\textbf{\textit{\{terminate, begin\}}} sein.\\

\textbf{Im Fehlerfall:}\\
Wird versucht eine Aktivität zu starten, die nicht existiert bzw. eine Aktivität zu beenden, die gar nicht gestartet war, wird einem ein Boolean mit dem Wert \textit{false} zurückgegeben.\\

2.2.\\ Wenn eine neue Instanz eines Szenarios gestartet werden soll, nutzt man den POST-Request mit folgender URL:\\
\textit{\textbf{http://172.16.64.113:8080/JEngine/Scenario/\\Start/\{ScenarioID\}}}\\

\textbf{Variable die benutzt wird:}\\
\textit{ScenarioID:} hier wird die ID des Scenarios erwartet. Dabei handelt es sich um einen Integer-Wert.\\

\textbf{Im Fehlerfall:}\\
Wird versucht ein Szenario zu starten, das nicht existiert, wird einem ein Integer mit dem Wert \textit{-1} zurückgegeben.\\

\end{enumerate}


Default URL:
http://172.16.64.113:8080/JEngine/interface/\{version\}/\{language\}/?

as parameter:
scenario
returns all scenarios
 
scenario=1
returns all instances

scenario=1\& runtime
returns all instances with runtime

instanceid=1\& status=enabled
returns all enabled activities

activityid=2
returns all information for this activity

users
returns all users

roles
roles returns all roles


timestamp ISO-8601

for more informations about the REST please have a look at the REST Specification..

%
%
\subsubsection{ExecutionService}
Der ExecutionService ist die Komponente, auf die die REST-API aufbaut. Diese Komponente der JEngine ist essentiell um Anfragen, die über das REST-Interface entgegengenommen werden, zu bearbeiten. Dabei stellt es eine Verbindung zur Datenbank her, um auf dieser Queries auszuführen. Die meisten Abfragen benötigen eine ID des Objekts, um mehr Informationen über dieses zu gewinnen. Es gibt zudem auch Methoden um Aktivitäten bzw. ganze Szenarien zu instanziieren.

%
%
\subsection{JComparser}

%
%
\subsection{JFrontEnd}

%
%
\subsection{JDatabase}

%
%
\section{Processeditor}

%
%
\section{PCM modelling using the Processeditor}
\label{pcm-modelling-using-the-processeditor}
This document explains how to use the Processeditor to create PCM models. A PCM-Process can be described by many PCM fragments and one PCM scenario.

%
%
\subsection{Preparations}
\label{preparations}
Currently you need both, the Processeditor Workbench and the Processeditor Server to model and Save PCM. You will use the Workbench for modelling and the Server as a global repository.

%
%
\subsection{PCM Fragments}
\label{pcm-fragments}
PCM Fragments are small Business Process models. They can be modelled using a subset of the BPMN-Notation:

\begin{itemize}
\itemsep1pt\parskip0pt\parsep0pt
\item
  Tasks
\item
  Events ** Blanko Start-Event ** Blanko End-Event
\item
  Gateways ** Parallel Gateway ** Exclusive Gateway
\item
  Data Objects
\item
  Sequence Flow
\item
  Data Flow
\end{itemize}

All this elements are offered by the model type PCM Fragment.

%
%
\subsubsection{Marking a Task as Global}
\label{marking-a-task-as-global}
PCM allows to use the same task in more than one fragment. To do so

\begin{enumerate}
\def\labelenumi{\arabic{enumi}.}
\itemsep1pt\parskip0pt\parsep0pt
\item
  model the Task (in one scenario)
\item
  Save the model to the repository
\item
  Right click on the Task and choose \emph{Properties}
\item
  Set the \emph{global flag}
\end{enumerate}

%
%
\subsubsection{Copy and Refer an existing Task}
\label{copy-and-refer-an-existing-task}

\begin{enumerate}
\def\labelenumi{\arabic{enumi}.}
\itemsep1pt\parskip0pt\parsep0pt
\item
  In another Fragment right click on any node
\item
  Choose ``Copy and Refer Task''
\item
  Connect to the server if necessary
\item
  Choose the Model and the Task you want to refer
\item
  Click on Ok
\end{enumerate}

%
%
\subsection{PCM Scenario}
\label{pcm-scenario}
A Scenario defines which PCM Fragments are part of one Process. All PCM
Fragments have to be saved on the Server. You can alter the Scenario
only by moving the nodes and adding/removing PCM Fragments.

%
%
\subsubsection{Defining a PCM Scenario}
\label{defining-a-pcm-scenario}

\begin{enumerate}
\def\labelenumi{\arabic{enumi}.}
\itemsep1pt\parskip0pt\parsep0pt
\item
  Create a new PCM Scenario Model.
\item
  Right Click on one of the two nodes
\item
  Choose Add Fragments
\item
  Mark all Models you want to add in the left List (CTRL for multi
  select)
\item
  click on add than on ok
\end{enumerate}

Now there should be entries for all the fragments (inside green node)
and for all their data objects (inside white node).

%
%
\subsubsection{Removing a Fragment From an Scenario}
\label{removing-a-fragment-from-an-scenario}

\begin{enumerate}
\def\labelenumi{\arabic{enumi}.}
\itemsep1pt\parskip0pt\parsep0pt
\item
  Right Click on one of the two nodes
\item
  Choose \emph{Add Fragments}
\item
  Select all the models you want to remove from the right list
\item
  Click on \emph{Remove} than click \emph{Ok}
\end{enumerate}

%
%
\subsubsection{Set a Termination Condition}
\label{set-a-termination-condition}

If a termination condition is full filled the process is terminated.
Currently only one termination condition consisting of one Data Object
in one specific state is possible.

\begin{enumerate}
\def\labelenumi{\arabic{enumi}.}
\itemsep1pt\parskip0pt\parsep0pt
\item
  Open your Scenario
\item
  Right Click on the canvas (not the Nodes)
\item
  Choose \emph{Properties}
\item
  Fill out the \emph{Termination Data Object} and \emph{Termination
  State} fields
\end{enumerate}

%
%
\subsubsection{Copy and Alter a Complete Fragment}
\label{copy-and-alter-a-complete-fragment}

You can create a variation of an existing PCM Fragment using the Plug-in
\emph{Create Variant}.

\begin{enumerate}
\def\labelenumi{\arabic{enumi}.}
\itemsep1pt\parskip0pt\parsep0pt
\item
  First click on \emph{Plug-Ins}
\item
  Choose \emph{Create Variant}
\item
  Choose your Fragment and click on \emph{Ok}
\end{enumerate}


%
%
\subsection{Processeditor Server}


%
%
\subsection{Processeditor Client}

%----------------------------------------------------------------------------------------
%	BIBLIOGRAPHY
%----------------------------------------------------------------------------------------

\bibliographystyle{apalike}

\bibliography{sample}

%----------------------------------------------------------------------------------------


\end{document}